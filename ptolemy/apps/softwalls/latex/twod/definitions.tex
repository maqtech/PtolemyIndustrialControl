% File giving definitions of commands and environments.
% Because of a latex bug with counters, this has to be included
% using \input not \include.

%--------------------------------------------------------------------------
% For the set of reals and integers
\newcommand{\rr}{\set{Reals}}
\newcommand{\ii}{\set{Integers}}
\newcommand{\cc}{\set{Complex}}
\newcommand{\nn}{\set{Naturals}}

%--------------------------------------------------------------------------
% For figure captions.
% Puts them in sanserif font, indented on both sides.
%   arg: The caption.
\newcommand{\figcaption}[1]{\textsf{\begin{center}\begin{minipage}{5in}
\caption{#1}
\end{minipage}\end{center}}}

%--------------------------------------------------------------------------
% For terms being defined.
% Puts them in bold face and creates an index entry.
%   arg: The term being defined.
% NOTE: To get boldface in the index, do |textbf after #1.
% But this breaks hyperlinks.
\newcommand{\defn}[1]{\textbf{#1}\index{#1}}

%--------------------------------------------------------------------------
% For terms being indexed.
% Puts them in standard font face and creates an index entry.
%   arg: The term being defined.
\newcommand{\pointer}[1]{#1\index{#1}}

%--------------------------------------------------------------------------
% For bold terms to be index, but defined elsewhere
% Puts them in bold face and creates an index entry.
%   arg: The term being defined.
\newcommand{\strong}[1]{\textbf{#1}\index{#1}}

%--------------------------------------------------------------------------
% For terms to be index, but defined elsewhere
% Puts them in normal face and creates an index entry.
%   arg: The term being defined.
\newcommand{\idx}[1]{#1\index{#1}}

%--------------------------------------------------------------------------
% For set names.
% Puts them in italics. In math mode, yields decent spacing.
%   arg: The name of the set.
\newcommand{\set}[1]{\mbox{\textit{#1}}}

%--------------------------------------------------------------------------
% For real part.
%   arg: The argument of the real part.
\newcommand{\re}[1]{\mbox{\textit{Re}}\{#1\}}

%--------------------------------------------------------------------------
% For imaginary part.
%   arg: The argument of the imaginary part.
\newcommand{\im}[1]{\mbox{\textit{Im}}\{#1\}}

%--------------------------------------------------------------------------
% For matlab commands
%   arg: The name of the command
\newcommand{\matlab}[1]{\texttt{#1}\index{#1 command in Matlab}\index{Matlab!#1}}
\newcommand{\simulink}[1]{\texttt{#1}\index{#1 in Simulink}\index{Simulink!#1}}
\newcommand{\matlabInCaption}[1]{\texttt{#1}}

%--------------------------------------------------------------------------
% For "Probing Further" sidebars.
% Puts them in a floating frame.  It is up to you to ensure that the
% frame fits on one page.
%   arg: the title.
\newenvironment{further}[1]{
\begin{table}[btp]
\centering
\begin{tabular}{|p{5in}|}
\hline
\cr
\begin{minipage} {5in}
\parskip        0.1in
\parindent      0.0in
\subsection* {Probing further: #1}
\addcontentsline{toc}{subsection}{Probing further: #1}
} {
\end{minipage}\cr
\cr
\hline
\end{tabular}
\end{table}
}

%--------------------------------------------------------------------------
% For "Basics" sidebars.
% Puts them in a floating frame.  It is up to you to ensure that the
% frame fits on one page.
%   arg: the title.
\newenvironment{basics}[1]{
\begin{table}[btp]
\centering
\begin{tabular}{|p{5in}|}
\hline
\cr
\begin{minipage} {5in}
\parskip        0.1in
\parindent      0.0in
\subsection* {Basics: #1}
\addcontentsline{toc}{subsection}{Basics: #1}
} {
\end{minipage}\cr
\cr
\hline
\end{tabular}
\end{table}
}

%--------------------------------------------------------------------------
% For "Tips and Tricks" sidebars.
% Puts them in a floating frame.  It is up to you to ensure that the
% frame fits on one page.
%   arg: the title.
\newenvironment{tricks}[1]{
\begin{table}[btp]
\centering
\begin{tabular}{|p{5in}|}
\hline
\cr
\begin{minipage} {5in}
\parskip        0.1in
\parindent      0.0in
\subsection* {Tips and Tricks: #1}
\addcontentsline{toc}{subsection}{Tips and Tricks: #1}
} {
\end{minipage}\cr
\cr
\hline
\end{tabular}
\end{table}
}

%--------------------------------------------------------------------------
% For text that is boxed for emphasis.
\newenvironment{boxed}{
\begin{center}
\begin{tabular}{|p{5in}|}
\hline
\cr
\begin{minipage} {5in}
\parskip        0.1in
\parindent      0.0in
} {
\end{minipage}\cr
\cr
\hline
\end{tabular}
\end{center}
}

%--------------------------------------------------------------------------
% For examples
% NOTE: Because of this line, this has to be included using \input
% not \include.
\newcounter{example}
\newenvironment{example}{
\refstepcounter{example}
\begin{quote}
\textbf{Example \arabic{chapter}.\arabic{example}:}
} {
\end{quote}
}
\renewcommand \theexample {\thechapter.\arabic{example}}
